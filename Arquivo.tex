\documentclass[12pt, a4paper]{article}

\usepackage[top=2cm, bottom=2cm, left=2.5cm, right=2.5cm]{geometry}
\usepackage{graphicx}
\usepackage[brazilian]{babel}
\usepackage[T1]{fontenc}
\usepackage[utf8]{inputenc}
%\usepackage{setspace}
\linespread{1.3}


\usepackage{color}
\definecolor{laranja}{RGB}{255, 165, 0}
\usepackage[dvipsnames]{xcolor}

\title{Acompanhamento do curso de LaTeX}
\author{Samuel \footnote{Graduado em EQ}}
%\date{13 de Dezembro de 2023}

\begin{document}
	\maketitle%\newpage
	
	\textcolor{laranja}{Texto com a cor laranja}
	
	{\color{Violet}{Texto com cor Aqua}}
	
	%\pagecolor{blue}
	
	\fcolorbox{red}{blue}{Texto na caixa}
	
	\begin{flushleft}
		Caros amigos, a valorização de fatores subjetivos nos obriga à análise de alternativas às soluções ortodoxas. Por outro lado, o entendimento das metas propostas cumpre um papel essencial na formulação dos conhecimentos estratégicos para atingir a excelência. Assim mesmo, o início da atividade geral de formação de atitudes estimula a padronização do sistema de participação geral. Podemos já vislumbrar o modo pelo qual a estrutura atual da organização apresenta tendências no sentido de aprovar a manutenção do orçamento setorial. Evidentemente, a contínua expansão de nossa atividade promove a alavancagem da gestão inovadora da qual fazemos parte.
	\end{flushleft}
	
	\begin{center}
		Neste sentido, a mobilidade dos capitais internacionais não pode mais se dissociar das direções preferenciais no sentido do progresso. Ainda assim, existem dúvidas a respeito de como a consulta aos diversos militantes desafia a capacidade de equalização do sistema de formação de quadros que corresponde às necessidades. As experiências acumuladas demonstram que o aumento do diálogo entre os diferentes setores produtivos obstaculiza a apreciação da importância das condições inegavelmente apropriadas. Acima de tudo, é fundamental ressaltar que o fenômeno da Internet faz parte de um processo de gerenciamento das condições financeiras e administrativas exigidas. O incentivo ao avanço tecnológico, assim como a adoção de políticas descentralizadoras acarreta um processo de reformulação e modernização das formas de ação.
	\end{center}
	
	\begin{flushright}
		Todas estas questões, devidamente ponderadas, levantam dúvidas sobre se a crescente influência da mídia prepara-nos para enfrentar situações atípicas decorrentes das diretrizes de desenvolvimento para o futuro. No entanto, não podemos esquecer que a consolidação das estruturas auxilia a preparação e a composição dos relacionamentos verticais entre as hierarquias. A certificação de metodologias que nos auxiliam a lidar com a execução dos pontos do programa ainda não demonstrou convincentemente que vai participar na mudança do retorno esperado a longo prazo. Desta maneira, o acompanhamento das preferências de consumo talvez venha a ressaltar a relatividade dos níveis de motivação departamental.
	\end{flushright}
	
	A prática cotidiana prova que a expansão dos mercados mundiais maximiza as possibilidades por conta de todos os recursos funcionais envolvidos. É claro que o desenvolvimento contínuo de distintas formas de atuação estende o alcance e a importância das posturas dos órgãos dirigentes com relação às suas atribuições. A nível organizacional, o desafiador cenário \\ globalizado garante a contribuição de um grupo importante na determinação dos paradigmas corporativos.
	
	O que temos que ter sempre em mente é que a necessidade de renovação processual possibilita uma melhor visão global dos índices pretendidos. Nunca é demais lembrar o peso e o significado destes problemas, uma vez que o consenso sobre a necessidade de qualificação agrega valor ao estabelecimento das novas proposições. O empenho em analisar a determinação clara de objetivos deve passar por modificações independentemente dos modos de operação convencionais. No mundo atual, a revolução dos costumes assume importantes posições no estabelecimento do impacto na agilidade decisória.

	É importante questionar o quanto a hegemonia do ambiente político é uma das consequências do levantamento das variáveis envolvidas. O cuidado em identificar pontos críticos no surgimento do comércio virtual pode nos levar a considerar a reestruturação do fluxo de informações. Do mesmo modo, o novo modelo estrutural aqui preconizado exige a precisão e a definição das diversas correntes de pensamento. Gostaria de enfatizar que o comprometimento entre as equipes representa uma abertura para a melhoria dos procedimentos normalmente adotados. Por conseguinte, a constante divulgação das informações causa impacto indireto na reavaliação das regras de conduta normativas.

	Não obstante, a competitividade nas transações comerciais oferece uma interessante oportunidade para verificação dos métodos utilizados na avaliação de resultados. Todavia, a complexidade dos estudos efetuados aponta para a melhoria do processo de comunicação como um todo. Percebemos, cada vez mais, que a percepção das dificuldades facilita a criação do investimento em reciclagem técnica. Pensando mais a longo prazo, o julgamento imparcial das eventualidades afeta positivamente a correta previsão do remanejamento dos quadros funcionais.
	
	Caros amigos, a revolução dos costumes prepara-nos para enfrentar situações atípicas decorrentes do remanejamento dos quadros funcionais. Por outro lado, o entendimento das metas propostas desafia a capacidade de equalização do processo de comunicação como um todo. Assim mesmo, o início da atividade geral de formação de atitudes estimula a padronização do investimento em reciclagem técnica. A certificação de metodologias que nos auxiliam a lidar com o acompanhamento das preferências de consumo assume importantes posições no estabelecimento dos paradigmas corporativos.
	
	Evidentemente, a consulta aos diversos militantes promove a alavancagem das regras de conduta normativas. O cuidado em identificar pontos críticos no julgamento imparcial das eventualidades representa uma abertura para a melhoria das direções preferenciais no sentido do progresso. Acima de tudo, é fundamental ressaltar que o consenso sobre a necessidade de qualificação estende o alcance e a importância do sistema de participação geral.
	
	Todas estas questões, devidamente ponderadas, levantam dúvidas sobre se o novo modelo estrutural aqui preconizado pode nos levar a considerar a reestruturação do orçamento setorial. Ainda assim, existem dúvidas a respeito de como a competitividade nas transações comerciais maximiza as possibilidades por conta das condições financeiras e administrativas exigidas. Por conseguinte, a estrutura atual da organização talvez venha a ressaltar a relatividade das formas de ação.
	
	Neste sentido, a percepção das dificuldades aponta para a melhoria das diretrizes de desenvolvimento para o futuro. O que temos que ter sempre em mente é que a consolidação das estruturas afeta positivamente a correta previsão dos procedimentos normalmente adotados. A nível organizacional, a execução dos pontos do programa ainda não demonstrou convincentemente que vai participar na mudança de alternativas às soluções ortodoxas. É claro que a complexidade dos estudos efetuados acarreta um processo de reformulação e modernização da gestão inovadora da qual fazemos parte.
	
	\section{Titulo da secção com numeração}
	
	Não obstante, a expansão dos mercados mundiais apresenta tendências no sentido de aprovar a manutenção de todos os recursos funcionais envolvidos. Desta maneira, a adoção de políticas descentralizadoras facilita a criação do impacto na agilidade decisória. Gostaria de enfatizar que a crescente influência da mídia exige a precisão e a definição das posturas dos órgãos dirigentes com relação às suas atribuições.
	
	\section*{Titulo da secção sem numeração}
	
	A prática cotidiana prova que a necessidade de renovação processual possibilita uma melhor visão global dos índices pretendidos. É importante questionar o quanto a contínua expansão de nossa atividade agrega valor ao estabelecimento das novas proposições. Pensando mais a longo prazo, a mobilidade dos capitais internacionais deve passar por modificações independentemente dos conhecimentos estratégicos para atingir a excelência.
	
	\begin{itemize}
		\item [$\otimes$] Primeiro ponto
		\item [$\otimes$] Segundo ponto
	\end{itemize}

	\begin{enumerate}
		\item O primeiro;
		\item O segundo.
	\end{enumerate}
	
	\begin{center}
		\rule{10cm}{0.02cm}\\
		Samuel Vitor Saraiva
		
		\vspace{1cm}
		
		\rule{10cm}{0.02cm}\\
		Samuel Vitor Saraiva
	\end{center}

	Eu vou me \hspace{3cm} separar daqui 
	
	
	Conteúdo 1
	
	%\hrulefill\\
	\dotfill\\
	
	Conteúdo 2
	
	
\end{document}